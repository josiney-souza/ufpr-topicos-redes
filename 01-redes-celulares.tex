\documentclass[11pt,oneside,a4paper]{abntex2}

\usepackage[brazil]{babel}
\usepackage[utf8]{inputenc}
\usepackage{url}
\usepackage{indentfirst}

\titulo{Uma revisão sobre redes móveis de celulares}
\autor{Josiney de Souza}

\begin{document}
\bibliographystyle{abnt-num}

\maketitle

\section*{Introdução}
\label{intro}

%Cebolas alhos. E afanar as minhas três Terezas. Eu tenho que pegar teu bruto. De que me agrade minto. Sorris a usar dentes. Para príncipe da condessa. Que a senhora tenha a cerveja. Sorris a usar dentes. Para príncipe da condessa. Por isso, quero lhe dê Zé.

\begin{itemize}
	\item Redes geral
	\item Tipos de rede (cabo, sem fio)
	\item Redes sem fio
	\item Redes celulares
\end{itemize}

As redes de computadores revolucionaram o mundo em que vivemos. Através delas, é possível acessar e compartilhar informações estando fisicamente perto ou longe. Nesse sentido, é comum encontrar pessoas usando diferentes aparelhos para as mais diversas finalidades, como trabalho, negócios, educação, entretenimento e outros.

Atualmente, encontra-se aparelhos computacionais como \textit{desktops}/PCs, \textit{laptops}/notebooks, celulares \textit{smartphones} e \textit{tablets} que fazem uso dessas redes. Elas estão geralmente disponíveis em duas modalidades: redes cabeadas e redes sem fio. Alguns aparelhos podem usar mais de uma modalidade para suas comunicações.

No que diz respeito às redes sem fio, os principais representantes são o Bluetooth, o Wi-Fi e as redes móveis celulares. O Bluetooth, como uma rede de abrangência PAN (Personal Area Network), tem alcance reduzido em comparação às demais. O Wi-Fi, como uma rede de abrangência LAN/WLAN (Local Area Network/Wireless Local Area Network), tem alcance maior que o primeiro porém em uma área delimitada não tão grande. A rede móvel de celular, como uma rede de abrangência WAN/WWAN (Wide Area Network/Wireles Wide Area Network) \cite{wwan}, tem alcance maior que as anteriores.

Ao longo do tempo, as redes móveis de celulares foram evoluindo até o momento atual. A primeira geração, fazia uso apenas de voz em formato analógico. A segunda geração, ainda fazia uso de voz porém em formato digital. A terceira geração, começou a disponibilizar dados além da transmissão de voz. A quarta geração, teve ... . A quinta geração, a atual, começou a ter ... . O momento atual também é de discussões sobre mudanças na quinta geração para que uma possível sexta geração incremente serviços a esse tipo de rede.

O restante do documento está assim dividido: a próxima seção aborda conceitos importantes para o entendimento das diferentes gerações de telefonia móvel e redes de dados celulares. As seções seguintes tratam de aprofundamentos e curiosidades sobre cada uma das gerações, estando dispostas em ordem cronológica. Por último, apresenta-se um fechamento do texto e as referências bibliográficas utilizadas.

\section*{Conceitos Importantes}
\label{conceitos}

\begin{itemize}
	\item Celulas
	\item Handover/handoff
	\item Pontos em comum
	\item Aparelhos, antenas, centrais de comutação
	\item Canais
\end{itemize}

\begin{center}
\begin{tabular}{||c||c||c||}
	\hline
	\textbf{Letra no visor} & \textbf{Tecnologia} & \textbf{Geração} \\
	\hline
	\hline
	G & GPRS & 2,5G \\
	\hline
	E & EDGE & 2,5G \\
	\hline
	H & HSPA & 3G \\
	\hline
	H+ & HSPA & 3G \\
	\hline
	LTE & LTE & 4G \\
	\hline
	5G & 5G & 5G \\
	\hline
\end{tabular}
\end{center}

\begin{enumerate}
	\item[G] GPRS $\rightarrow$ 2,5G
	\item[E] EDGE $\rightarrow$ 2,5G
	\item[H] HSPA $\rightarrow$ 3G
	\item[H+] HSPA Plus $\rightarrow$ 3G
	\item[LTE] LTE $\rightarrow$ 4G
	\item[5G] 5G $\rightarrow$ 5G
\end{enumerate}

\section*{1G - Geração 1}
\label{1g}

Nos anos 1980, surgiu a primeira geração de rede móvel para a telefonia celular através da American Telephone and Telegraph (AT\&T) e foi chamada de Advanced Mobile Phone Service (AMPS) \cite{tcc1}. Como não havia um padrão definido para os diferentes países, na mesma época haviam sistemas concorrentes, como o da Nippon Telegraph and Telephone (NTT) no Japão; o da Nordic Mobile Telephone (NMT) nos países nórdicos; e o da Total Access Communication System (TACS) no Reino Unido; além de outros \cite{wiki-1g-br, wiki-1g-en}. No Brasil, foi adotado o padrão AMPS \cite{tcc2}.

Conhecida como 1G, essa primeira geração teve um desenvolvimento lento. Nas Américas, após o lançamento da tecnologia, as expansões para os demais países tiveram intervalos de um a dois anos \cite{repor1}. Tecnicamente, usava multiplexação de acesso FDMA; células com diâmetros de 10 km a 20 km \cite{aula3}; voz em formato analógico e faixa de frequência da ordem de 800 MHz.

Para fazer uma ligação, o usuário deve digitar o número desejado e apertar em ``chamar''. O celular então envia para a estação base seu próprio número e também o do celular de destino. A estação base, por sua vez, envia os dados recebidos para a central de comutação. Essa última escolhe uma frequência livre disponível para a chamada e retorna para o telefone.

Na outra ponta, o destino deve ficar em modo de escuta. A central de comutação, ao receber o pedido de comunicação, repassa a necessidade para a estação base do destino; e esta repassa para o telefone chamado  usando o canal de \textit{paging}. O destino retorna o aceite através do canal de acesso e a estação base informa o canal da ligação, para só então o telefone tocar.

\section*{2G - Geração 2}
\label{2g}

Na década de 90 e anos 2000, surgiu a segunda geração de rede móvel para a telefonia celular, que permitiu não só a transmissão de voz em formato digital como também a transferência de dados. É nessa época que surgem o Short Messages Service (SMS), o cartão de memória Subscriber Identity Module (SIM) e a identificação International Mobile Equipment Identity (IMEI); os dois últimos diretamente ligados à segurança da comunicação por voz e dados através de criptografia \cite{tcc2}.

Nessa segunda geração, a tecnologia anterior AMPS evoluiu para Digital Advanced Mobile Phone System (D-AMPS) e surgiu o Global System for Mobile Communications (GSM) como uma tentativa de padronização, sendo adotado no mundo todo \cite{aula3}. No primeiro, como era uma evolução do AMPS, possuía compatibilidade com a tecnologia do 1G e continuava a usar o FDMA; e trouxe como novidade o TDMA, sendo esse nome associado ao D-AMPS. No GSM, usava acesso múltiplo via uma combinação de TDMA e FDMA. No Brasil, foi adotado primeiramente o padrão D-AMPS/TDMA e substituído pelo GSM em 2002 \cite{tcc2}.

Os equipamentos envolvidos em uma comunicação geralmente compreendem:
\begin{description}
	\item[User Equipment (UE) / Mobile Station (MS)] aparelho celular do usuário;
	\item[Base Transceiver Station (BTS)] antena/torre/Estação de Rádio Base (ERB) da operadora;
	\item[Base Station Controller (BSC)] controlador;
	\item[Mobile Switching Center (MSC)] comutadora/central de comutação;
	\item[Gateway Mobile Switching Center (GMSC)] ponto de comutação entre a rede celular e a rede de telefonia fixa PSTN (Public Switched Telephone Network).
\end{description}

Há algumas bases de dados envolvidas nessas comunicações:
\begin{description}
	\item[Home Location Register (HLR)] assinantes locais;
	\item[Visitor Location Register (VLR)] celulares temporários na célula;
	\item[Equipment Identity Register (EIR)] lista de identificadores dos aparelhos;
	\item[AUthentication Center (AuC)] para autenticar assinantes.
\end{description}

Na comunicação, o aparelho celular do usuário se comunica com a antena. As antenas tanto transmitem para o celular quanto se ligam ao controlador. Os controladores controlam as diversas antenas, suas células, o handoff dos usuários e se comunicam com a comutadora. As comutadoras formam uma rede entre si e também possuem saída para outras redes.

De forma geral, a rede GSM é dividida em duas partes distintas: a Rede de Acesso e o Núcleo da Rede. A Rede de Acesso, também chamada de Radio Access Network (RAN) ou de Base Station System (BSS), é composta das antenas e da controladora. O Núcleo da Rede, também chamado de Core Network ou de Switching System, é composto pela comutadora e pelas bases de dados.

\subsection*{2,5G - Geração 2,5}

Nos anos 2000, foram propostas duas evoluções para a rede GSM através de dois padrões que focavam na transmissão de dados: General Packet Radio Service (GPRS) e a Enhanced Data Rates for GSM Evolution (EDGE). Essas duas propostas ficaram conhecidas como geração 2,5 ou 2,5G e funcionavam sobre a base do GSM adicionando novos hardware e controles. Em ambos padrões, a voz nas comunicações ocorriam por comutação de pacotes, enquanto que o GSM sem esses padrões fazia uso da comutação de circuitos \cite{tcc2}.

\section*{3G - Geração 3}
\label{3g}

Apresentei a pistola. Apresentei a pistola. Escrevi porque quero tonto. Que a senhora tenha a cerveja. Que a senhora tenha a cerveja. Sorris a usar dentes. Eu tenho que pegar teu bruto. E afanar as minhas três Terezas. E afanar as minhas três Terezas. E afanar as minhas três Terezas.

\section*{4G - Geração 4}
\label{4g}

Eu tenho que pegar teu bruto. Escrevi porque quero tonto. Cebolas alhos. De que me agrade minto. De que me agrade minto. Por isso, quero lhe dê Zé. E afanar as minhas três Terezas. Por isso, quero lhe dê Zé. De que me agrade minto. Que a senhora tenha a cerveja.

\section*{5G - Geração 5}
\label{5g}

Apresentei a pistola. Sorris a usar dentes. De que me agrade minto. Apresentei a pistola. Apresentei a pistola. Para príncipe da condessa. Que a senhora tenha a cerveja. E afanar as minhas três Terezas. Em sua beca. De que me agrade minto.

\section*{6G - Geração 6}
\label{6g}

Cebolas alhos. Sorris a usar dentes. Cebolas alhos. Cebolas alhos. Cebolas alhos. E afanar as minhas três Terezas. Sorris a usar dentes. Eu tenho que pegar teu bruto. E afanar as minhas três Terezas. E afanar as minhas três Terezas.

\section*{Conclusão}
\label{conclusao}

Escrevi porque quero tonto. Apresentei a pistola. Eu tenho que pegar teu bruto. Eu tenho que pegar teu bruto. Eu tenho que pegar teu bruto. Sorris a usar dentes. Cebolas alhos. Por isso, quero lhe dê Zé. E afanar as minhas três Terezas. E afanar as minhas três Terezas.

\bibliography{bibliografias}
\label{biblio}

\end{document}

