\documentclass[11pt,oneside,a4paper]{abntex2}

\usepackage[brazil]{babel}
\makeatletter\AtBeginDocument{\let\@elt\relax}\makeatother
\usepackage[utf8]{inputenc}
\usepackage{url}
\usepackage{indentfirst}
\usepackage{multirow}
\usepackage{longtable}

\titulo{Uma revisão sobre redes móveis de celulares}
\autor{Josiney de Souza}

\begin{document}
\bibliographystyle{abnt-num}

\maketitle

\section*{Introdução}
\label{intro}

As redes de computadores revolucionaram o mundo em que vivemos. Através delas, é possível acessar e compartilhar informações estando fisicamente perto ou longe. Nesse sentido, é comum encontrar pessoas usando diferentes aparelhos para as mais diversas finalidades, como trabalho, negócios, educação, entretenimento e outros.

Atualmente, encontra-se aparelhos computacionais como \textit{desktops}/PCs, \textit{laptops}/notebooks, celulares \textit{smartphones} e \textit{tablets} que fazem uso dessas redes. Elas estão geralmente disponíveis em duas modalidades: redes cabeadas e redes sem fio. Alguns aparelhos podem usar mais de uma modalidade para suas comunicações.

No que diz respeito às redes sem fio, os principais representantes são o Bluetooth, o Wi-Fi e as redes móveis celulares. O Bluetooth, como uma rede de abrangência PAN (Personal Area Network), tem alcance reduzido em comparação às demais. O Wi-Fi, como uma rede de abrangência LAN/WLAN (Local Area Network/Wireless Local Area Network), tem alcance maior que o primeiro porém em uma área delimitada não tão grande. A rede móvel de celular, como uma rede de abrangência WAN/WWAN (Wide Area Network/Wireless Wide Area Network) \cite{wwan}, tem alcance maior que as anteriores.

Ao longo do tempo, as redes móveis de celulares foram evoluindo até o momento atual. A primeira geração, fazia uso apenas de voz em formato analógico. A segunda geração, ainda fazia uso de voz porém em formato digital. A terceira geração, começou a disponibilizar dados além da transmissão de voz. A quarta geração, teve um aumento das taxas de transmissão e popularização do acesso à Internet pelos dispositivos. A quinta geração, a atual, começou a ter a separação de planos (de controle e de usuário), a definição de funções por software e uma maior ampliação de taxas e acessos. O momento atual também é de discussões sobre mudanças na quinta geração para que uma possível sexta geração incremente serviços a esse tipo de rede.

O restante do documento está assim dividido: a próxima seção aborda conceitos importantes para o entendimento das diferentes gerações de telefonia móvel e redes de dados celulares. As seções seguintes tratam de aprofundamentos e curiosidades sobre cada uma das gerações, estando dispostas em ordem cronológica. Por último, apresenta-se um fechamento do texto e as referências bibliográficas utilizadas.

\section*{Conceitos Importantes}
\label{conceitos}

A rede de telefonia móvel celular tem alguns conceitos importantes necessários de serem entendidos e definidos antes de se compreender as gerações e os termos usados em cada uma delas. Esses conceitos envolvem a definição de células, handover/handoff, canais, os equipamentos que fazem parte da comunicação e a estrutura/organização de equipamentos.

As células, que dão o nome para a telefonia celular, são as áreas geográficas cobertas pelos sinais de rádio frequência que as antenas/torres das operadoras emitem para os aparelhos celulares \cite{def-celula}. Cada aparelho celular, independente da tecnologia ou geração, está ligado a uma célula. Em áreas com menos densidade populacional, o tamanho da célula é grande enquanto que, em regiões com maior densidade populacional, as células são menores para que possa haver mais células e atender um quantitativo maior \cite{livro-comer}.

O handoff ou handouver (também chamado de transferência) é quando um usuário ou equipamento se move para além de sua célula atual e necessita que outra célula vizinha assuma a responsabilidade pela comunicação e controle do aparelho \cite{def-handoff}. O handoff também pode acontecer quando o sinal da atual célula não está atendendo às condições mínimas para a continuação da comunicação.

A comunicação celular faz uso de alguns canais de comunicação. Em algumas tecnologias, são usados canais separados para recepção (\textit{downlink}) e para transmissão (\textit{uplink}). Esses canais podem ser físicos ou lógicos. Os canais também podem ser de tráfego/transporte ou de controle \cite{def-canal, tcc2}. Um dos canais de controle é o canal de \textit{paging}, usado para notificações aos equipamentos dos usuários, por exemplo, uma chamada que chega \cite{def-paging}.

\subsection*{Comunicação}

Os equipamentos envolvidos em uma comunicação geralmente compreendem:
\begin{itemize}
	\item aparelho celular do usuário;
	\item antena/torre/Estação de Rádio Base (ERB) da operadora;
	\item controlador;
	\item comutadora/central de comutação;
	\item ponto de comutação entre a rede celular e a rede de telefonia fixa PSTN (Public Switched Telephone Network), Internet ou outras redes.
\end{itemize}

Há algumas bases de dados envolvidas nessas comunicações:
\begin{itemize}
	\item assinantes locais;
	\item celulares temporários na célula (visitantes);
	\item lista de identificadores dos aparelhos;
	\item para autenticar assinantes.
\end{itemize}

Na comunicação, o aparelho celular do usuário se comunica com a antena. As antenas tanto transmitem para o celular quanto se ligam ao controlador. Os controladores controlam as diversas antenas, suas células, o handoff dos usuários e se comunicam com a comutadora. As comutadoras formam uma rede entre si e também possuem saída para outras redes.

\subsection*{Estrutura}

De forma geral, a rede móvel de telefonia celular é dividida em duas partes distintas: a Rede de Acesso e o Núcleo da Rede. A Rede de Acesso, também chamada de Radio Access Network (RAN) ou de Base Station System (BSS) na 2G, Universal Mobile Telecommunications System (UMTS) Terrestrial RAN (UTRAN) na 3G, Evolved Universal Terrestrial RAN (E-UTRAN) na 4G, é composta das antenas e da controladora. Na 1G ela não era uma entidade isolada e na 5G existem os esforços para tornar a RAN aberta e dividida em serviços em software, sendo chamada de Open RAN ou O-RAN.

O Núcleo da Rede, também chamado de Core Network ou especificamente de Switching System (SS) na 2G e de Evolved Packet Core (EPC) na 4G, é composto pela comutadora e pelas bases de dados. Assim como para a RAN, na 1G não era uma entidade isolada e na 5G os esforços são para deixá-la como serviços em software.

Abaixo, na Tabela \ref{sopa-letrinhas-1}, encontra-se um compilado de nomes de equipamentos e seus correspondentes em cada uma das gerações celulares.

\begin{center}
\begin{longtable}{|m{1.5cm}|m{2.5cm}|m{2cm}|m{2cm}|m{2cm}|m{2cm}|m{2cm}|}
\caption{Nomes dos equipamentos de telefonia móvel celular em cada geração}
\label{sopa-letrinhas-1}\\
\hline
                        & Equipamentos                           & 1G & 2G                                       & 3G                                                                                          & 4G                                                              & 5G                                                                                                                                          \\ \hline
\endfirsthead
%
\multicolumn{7}{c}%
{{\bfseries Tabela \thetable\ continuada da página anterior}} \\
\hline
                        & Equipamentos                           & 1G & 2G                                       & 3G                                                                                          & 4G                                                              & 5G                                                                                                                                          \\ \hline
\endhead
%
\multirow{3}{*}{RAN}    & Aparelho Celular / Dispositivo do Usuário        & User Equipment (UE)                                                                           & User Equipment (UE) & User Equipment (UE)                                                                         & User Equipment (UE)                                             & User Equipment (UE)                                                                                                                         \\ \cline{2-7} 
                        & Estação de Rádio Base (ERB)                 &   Base Station & Base Transceiver Station (BTS)           & NodeB                                                                                       & eNodeB                                                          & new Generation NodeB (gNB); Unidade de Rádio; Unidade Distribuída; Unidade Central                                                          \\ \cline{2-7} 
                        & Controlador                            & Mobile Transport Network   & Base Station Controller (BSC)            & Radio Network Controller (RNC)                                                              & Mobility Management Entity (MME)                                & Access and Mobility Management Function (AMF)                                                                                               \\ \hline
\multirow{2}{*}{Núcleo} & Comutadora / Central de Comutação      & Mobile Transport Network   & Mobile Switching Center (MSC)            & Media GateWay (MGW); Serving GPRS Support Node (SGSN); Mobile Switching Center Server (MSC) & Policy \& Charging Rules Function (PCRF); Serving Gateway (SGW) & Policy Control Function (PCF); Network Exposure Function (NEF); Network Repository Function (NRF); Network Slicing Selector Function (NSSF) \\ \cline{2-7} 
                        & Gateway                                & Local Exchange   & Gateway Mobile Switching Center (GMSC)   & Gateway GPRS Support Node (GGSN)                                                            & Packet Data Network Gateway (PGW)                               & User Plane Function (UPF); Session Management Function (SMF)                                                                                \\ \hline
\multirow{4}{*}{Bases}  & Assinantes Locais                      & Mobile Transport Network   & Home Location Register (HLR)             & Home Location Register (HLR)                                                                & Home Subscriber Server (HSS)                                    & Structured Data Storage Network Function (SDSF)                                                                                             \\ \cline{2-7} 
                        & Assinantes Visitantes                  & Mobile Transport Network   & Visitor Location Register (VLR)          & Visitor Location Register (VLR)                                                             & Home Subscriber Server (HSS)                                    & Unstructured Data Storage Network Function (UDSF)                                                                                           \\ \cline{2-7} 
                        & Lista de identificadores dos aparelhos & Mobile Transport Network   & Equipment Identity Register (EIR)        &                                                                                             & Home Subscriber Server (HSS)                                    & Unified Data Management (UDM)                                                                                                               \\ \cline{2-7} 
                        & Autenticador                           & Mobile Transport Network   & AUthentication Center (AuC)              &                                                                                             & Home Subscriber Server (HSS)                                    & Authentication Server Function (AUSF)                                                                                                       \\ \hline
\end{longtable}
\end{center}

\section*{1G - Geração 1}
\label{1g}

Nos anos 1980, surgiu a primeira geração de rede móvel para a telefonia celular através da American Telephone and Telegraph (AT\&T) e foi chamada de Advanced Mobile Phone Service (AMPS) \cite{tcc1}. Como não havia um padrão definido para os diferentes países, na mesma época haviam sistemas concorrentes, como o da Nippon Telegraph and Telephone (NTT) no Japão; o da Nordic Mobile Telephone (NMT) nos países nórdicos; e o da Total Access Communication System (TACS) no Reino Unido; além de outros \cite{wiki-1g-br, wiki-1g-en}. No Brasil, foi adotado o padrão AMPS \cite{tcc2}.

Conhecida como 1G, essa primeira geração teve um desenvolvimento lento. Nas Américas, após o lançamento da tecnologia, as expansões para os demais países tiveram intervalos de um a dois anos \cite{repor1}. Tecnicamente, usava multiplexação de acesso FDMA; células com diâmetros de 10 km a 20 km \cite{aula3}; voz em formato analógico e faixa de frequência da ordem de 800 MHz.

Para fazer uma ligação, o usuário deve digitar o número desejado e apertar em ``chamar''. O celular então envia para a estação base seu próprio número e também o do celular de destino. A estação base, por sua vez, envia os dados recebidos para a central de comutação. Essa última escolhe uma frequência livre disponível para a chamada e retorna para o telefone.

Na outra ponta, o destino deve ficar em modo de escuta. A central de comutação, ao receber o pedido de comunicação, repassa a necessidade para a estação base do destino; e esta repassa para o telefone chamado  usando o canal de \textit{paging}. O destino retorna o aceite através do canal de acesso e a estação base informa o canal da ligação, para só então o telefone tocar.

\section*{2G - Geração 2}
\label{2g}

Na década de 90 e anos 2000, surgiu a segunda geração de rede móvel para a telefonia celular, que permitiu a transmissão de voz em formato digital. É nessa época que surgem o Short Messages Service (SMS), o cartão de memória Subscriber Identity Module (SIM) e a identificação International Mobile Equipment Identity (IMEI); os dois últimos diretamente ligados à segurança da comunicação por voz e dados através de criptografia \cite{tcc2}.

Nessa segunda geração, a tecnologia anterior AMPS evoluiu para Digital Advanced Mobile Phone System (D-AMPS) e surgiu o Global System for Mobile Communications (GSM) como uma tentativa de padronização, sendo adotado no mundo todo \cite{aula3}. No primeiro, como era uma evolução do AMPS, possuía compatibilidade com a tecnologia do 1G e continuava a usar o FDMA; e trouxe como novidade o TDMA, sendo esse nome associado ao D-AMPS. No GSM, usava acesso múltiplo via uma combinação de TDMA e FDMA. No Brasil, foi adotado primeiramente o padrão D-AMPS/TDMA e substituído pelo GSM em 2002 \cite{tcc2}.

\subsection*{2,5G - Geração 2,5}

Nos anos 2000, foram propostas duas evoluções para a rede GSM através de dois padrões que focavam na transmissão de dados: General Packet Radio Service (GPRS) e a Enhanced Data Rates for GSM Evolution (EDGE). Essas duas propostas ficaram conhecidas como geração 2,5 ou 2,5G e funcionavam sobre a base do GSM adicionando novos hardware e controles. Em ambos padrões, a voz nas comunicações ocorriam por comutação de pacotes, enquanto que o GSM sem esses padrões fazia uso da comutação de circuitos \cite{tcc2}.

\section*{3G - Geração 3}
\label{3g}

Em 2008, se consolidou a terceira geração de rede móvel para a telefonia celular, que permitiu a transmissão não só de voz em formato digital como também a de dados. A partir de então, foi possível realizar videoconferências, downloads de vídeos, jogos interativos e Voz sobre Internet Protocol (IP) \cite{tcc1} e ainda manter compatibilidade com as redes de gerações anteriores.

Diferente das anteriores, esta geração foi bastante influenciada pelo padrão International Mobile Telecommunications 2000 (IMT-2000) da International Telecommunication Union (ITU). A partir dele, as tecnologias Universal Mobile Telecommunications System (UMTS), primeiramente mais utilizada na União Europeia, e o CDMA-2000, inicialmente mais usado nos Estados Unidos, foram desenvolvidas sob o acompanhamento da 3rd Generation Partnership Project (3GPP).

Nessas tecnologias, a divisão de acesso múltiplo é feita com CDMA, que usa toda a faixa de frequência disponível para a transmissão (sem divisão de frequência como no FDMA) e sem intervalos de tempos específicos para a transmissão (ao contrário do TDMA). É possível fazer transmissões massivas sem colisões entre os transmissores em razão de  uma estrutura chamada ``chips'', onde cada bit é representado por uma determinada quantidade de chips, geralmente 128.

Cada dispositivo tem sua sequência de chips definida como um padrão de fábrica e as sequências são ortogonais entre si. Caso queira transmitir o equivalente ao bit 1, transmite sua própria sequência de chips; caso contrário, se quiser transmitir o bit 0, transmite o complemento da sua própria sequência de chips. A partir dessa técnica, pode-se considerar algumas propriedades: o produto interno normalizado de duas sequências quaisquer é 0; o produto interno normalizado de uma sequência por ela mesmo é 1; o produto interno normalizado de uma sequência pelo seu complemento é -1 \cite{aula4}.

Em um determinado momento, a informação que chega para os dispositivos é uma sequência S que representa a soma de chips de todos os aparalhos que realizaram transmissão. Considerando-se as propriedades, para se recuperar a informação transmitida, faz-se o produto interno normalizado entre a sequência recebida e sequência ortogonal de chips do dispositivo D transmissor de quem se quer descobrir o bit transmitido.

\section*{4G - Geração 4}
\label{4g}

A quarta geração de rede móvel para a telefonia celular, que teve seu início na época de 2010, também se constituiu como uma rede IP. Isso significa que foi possível agregar segurança, disponibilizar um número maior de dispositivos conectados com acesso à Internet, aumentar as taxas de transmissões e melhorar as ofertas de serviços iniciadas na rede 3G.

Essas melhorias foram possíveis devido à sua comutação de pacotes, tendo sua rede de acesso, a Evolved UMTS Terrestrial Radio Access Network (E-UTRAN), e o núcleo da rede, o Evolved Packet Core (EPC), sido projetados para isso sobre os padrões da 3GPP. Quanto ao EPC da rede 4G, ele é retrocompatível com os núcleos das redes 3G e 2G. No 4G, as tecnologias que se destacaram foram a International Mobile Telecommunications-Advanced (IMT Advanced), a Long Term Evolution (LTE) \cite{aula5} e a WIMAX. A tecnologia LTE é o padrão predominante no Brasil \cite{tcc2}.

Uma das características na rede 4G é que o sinal de celular pode percorrer múltiplos caminhos até chegar ao destino, seja através da linha de visada ou via reflexões. Isso pode levar a múltiplas recepções do sinal. Esse problema é solucionado com o Channel Quality Indicator (CQI), onde há uma comunicação periódica com a ERB.

Além dos múltiplos caminhos, o 4G trouxe a multipropagação com o Multiple-Input Multiple-Output (MIMO); ``uma tecnologia de radiotransmissão que usa múltiplos transmissores e antenas para melhorar o uso do espectro disponível'' \cite{aula5}. Essa tecnologia também foi incorporada em outras redes sem fio, como Wi-Fi 4, Wi-Fi 5, HSPA+ (3G) e 5G \cite{mimo} para aumentar suas taxas de transferências \cite{tcc2}.

\section*{5G - Geração 5}
\label{5g}

A quinta geração de rede móvel para a telefonia celular, conhecida como New Gereration (NG), atualmente em uso e sendo expandida, objetiva conectar ainda mais dispositivos e melhorar suas taxas de transmissão. Assim, permite aplicações do tipo Internet das Coisas (Internet of Things - IoT), carros autônomos e casas inteligentes \cite{tcc2}, além de sistemas de missão crítica, realidade aumentada e realidade virtual \cite{aula6}. Nesta geração, a 3GPP também é a organização responsável pelos padrões adotados.

Assim como ocorria na rede 4G, na rede 5G o sinal de celular também pode percorrer múltiplos caminhos até chegar ao destino e levar a múltiplas recepções do sinal; então esta rede também faz uso do CQI. A novidade, visando o aprimoramento da rede, é o uso do QoS Class Identifier (QCI) para indicar classes e prioridades diferentes, dependendo da
necessidade das aplicações. Em conjunto com o CQI e outros fatores,  essas escolhas determinam as tecnologias de transmissão mais adequadas para cada aparelho.

Uma das características das redes 5G é a separação de planos, havendo o plano de controle e o plano de usuário no chamado Control and User Plane Separation (CUPS). No plano de usuário, os equipamentos se focam nas transmissões e recepções das informações trafegadas, sejam na rede da operadora ou em outra. No plano de controle, são realizados os controles dos aparelhos como registro, autenticação e mobilidade.

A tendência nas redes 5G é a separação de partes, controles e funções; tanto na RAN (chamada de NG-RAN) quanto no núcleo da rede (chamado de NG-Core). Na NG-RAN, há movimentos para a separação de funções que antes esta assumia inteiramente e que se torne um padrão aberto e definido por software. Isso permite que a NG-RAN tenha um controlador inteligente e possa oferecer microsserviços, contribuindo com a Edge Cloud e a computação em névoa.

Para o NG-Core, ele é organizado em três grupos de módulos: o primeiro, no plano de controle alinhado a componentes 4G; o segundo, também no plano de controle e alinhado a outras redes; e o terceiro, no plano de usuário para, dentre outras funções, encaminhar o tráfego entre a RAN e a Internet.

\section*{Sumarização}
\label{sumarização}

Nesta seção, faz-se uma sumarização das diferentes informações apresentadas no texto e outras informações pesquisadas em livros e sites da Internet. O objetivo é mostrar um comparativo entre as diferentes gerações da telefonia móvel celular e também um guia simples para identificar as tecnologias usadas e as gerações que estão em uso quando alguma determinada letra aparece junto com o símbolo da torre de celular nos aparelhos, especialmente quando se usam dados móveis nos aparelhos.

\subsection*{Comparativo entre gerações}

A seguir, na Tabela \ref{sumari}, é possível verificar um comparativo de informações sobre todas as gerações celulares.

\begin{center}
\begin{longtable}{|m{1.6cm}|m{1.6cm}|m{1.6cm}|m{1.6cm}|m{1.6cm}|m{1.6cm}|m{1.6cm}|m{1.6cm}|}
\caption{Comparativo de características entre as diferentes gerações}
\label{sumari}\\
\hline
                            & 1G                                                                                  & 2G                          & 2,5G                         & 3G                                                       & 4G                                               & 5G                                & 6G                 \\ \hline
\endfirsthead
%
\multicolumn{8}{c}%
{{\bfseries Tabela \thetable\ continuada da página anterior}} \\
\hline
                            & 1G                                                                                  & 2G                          & 2,5G                         & 3G                                                       & 4G                                               & 5G                                & 6G                 \\ \hline
\endhead
%
Época                       & 1980                                                                                & 1990                        & 2000                         & 2008                                                     & 2010                                             & 2020                              & ? (2028?)          \\ \hline
Padrões                     & AMPS (Américas); NTT (Japão); NMT (países nórdicos); TACS (Reino Unido); Outros ... & D-AMPS; GSM                 & GPRS; EDGE                   & WCDMA/UMTS (UE); CDMA-2000 (EUA)                         & IMT Advanced; LTE                                & New Radio (NR)                    & ?                  \\ \hline
Acesso múltiplo             & FDMA                                                                                & FDMA; TDMA                  & FDMA; TDMA                   & CDMA                                                     & OFDMA                                            & OFDMA                             & ?                  \\ \hline
Faixa de frequência         & 800 MHz                                                                             & 800 MHz; 900 MHz a 1900 MHz & 800 MHz; 900 MHz a 1900 MHz  & 1900 MHz                                                 & 700 MHz; 850 MHz; 1.800 MHz; 2.100 MHz; 2600 MHz & 700 MHz; 2,3 GHz; 3,5 GHz; 26 GHz & de 95 GHz a 3 THz \\ \hline
Taxa de transmissão teórica & 2.4 kb/s                                                                            & 15 kb/s                     & 115 kb/s; 384 kb/s           & 84 Mb/s                                                  & 100 Mb/s (download); 50 Mb/s (upload)            & 20 Gb/s                           & 1 Tbps             \\ \hline
Taxa de transmissão real    & 2.4 kb/s                                                                            & 15 kb/s                     & 30 - 40 kb/s; 110 - 120 kb/s & 2 Mb/s (parado); 384 kb/s (pedestre); 144 kb/s (veículo) & 5 Mb/s                                           & 100 Mb/s a 400 Mb/s               & ?                  \\ \hline
\end{longtable}
\end{center}

\subsection*{Tecnologias}

A seguir, na Tabela \ref{sopa-letrinhas-2}, é possível observar a relação entre as tecnologias, as gerações e as letras de identificação que aparecem no visor dos aparelhos celulares.

\begin{table}[htb!]
\caption{Relação entre letras que aparecem no visor do celular, tecnologias e gerações}
\label{sopa-letrinhas-2}
\begin{center}
\begin{tabular}{||c||c||c||}
	\hline
	\textbf{Letra no visor} & \textbf{Tecnologia} & \textbf{Geração} \\
	\hline
	\hline
	G & GPRS & 2,5G \\
	\hline
	E & EDGE & 2,5G \\
	\hline
	H & HSPA & 3G \\
	\hline
	H+ & HSPA+ & 3G \\
	\hline
	LTE & LTE & 4G \\
	\hline
	5G & 5G & 5G \\
	\hline
\end{tabular}
\end{center}
\legend{Fonte: o autor (2023)}
\end{table}

\section*{Conclusão}
\label{conclusao}

As redes de telefonia móvel revolucionaram o modo como vivemos. Nas primeiras gerações, era possível apenas se comunicar através de voz e/ou via SMS. Nas gerações mais novas, os dados se tornaram tão ou mais importantes que as ligações. Porém, em todas elas, a comunicação se dá de um mesmo \textit{modus operandi}: a partir do dispositivo do usuário, esse se comunica com a rede de acesso (RAN) inicialmente via estações de rádio base. A partir de então, no núcleo da rede, as informações ou conexões passam pelas controladoras, comutadoras, acessam as bases de informações e se dirigem aos dispositivos de destino, seja na própria rede ou em outras, como a Internet.

Atualmente, vivemos a transição da quarta para a quinta geração de telefonia móvel que objetiva aumentar as taxas de transmissão, melhorar os serviços já existentes e permitir com que mais dispositivos façam uso da tecnologia. Também temos os incentivos das instituições globais em criar módulos abertos na RAN ou no núcleo da rede. Ainda, é o momento das organizações mundiais pensarem a próxima geração da telefonia móvel, se receberá um novo nome/número ou se será um \textit{release} sobre a 5G.

Este trabalho tentou fazer uma consolidação de informações sobre as redes de telefonia móvel. Inicialmente, trouxe conceitos importantes para o entendimento do trabalho como um todo e definições que são comuns a todo o histórico dessas redes; a exemplo da maneira como a comunicação ocorre, os equipamentos envolvidos nessas comunicações e o comparativo da estrutura das redes ao longo das gerações.

Após, complementando as definições comuns, mostrou-se as principais informações das diferentes gerações de telefonia móvel de maneira cronológica, desde a 1G até a 5G. Depois apresentou-se uma sumarização dos conceitos trabalhados, sendo um quadro comparativo entre as gerações e um quadro comparativo da relação entre as tecnologias usadas, as gerações e as letras que aparecem no visor dos aparelhos celulares.

\bibliography{bibliografias}
\label{biblio}

\end{document}

