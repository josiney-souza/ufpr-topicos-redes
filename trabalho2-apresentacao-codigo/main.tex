\documentclass[xcolor=dvipsnames,table]{beamer}
\useinnertheme{rounded}
\useoutertheme{infolines}

\usepackage[portuguese]{babel}
\usepackage[utf8]{inputenc}
\usepackage{url}
\usepackage{multicol}

\title{Aplicação cliente-servidor baseada em KVS com TLS\\ Código-fonte}
\subtitle{Tópicos em Redes de Computadores (INFO-7065)}
\author{Josiney de Souza (josiney.souza@ifc.edu.br)}
\institute{UFPR / DInf}
\date{\today}

%%% Apagar barra de navegacao:
% https://stackoverflow.com/questions/3210205/how-to-get-rid-of-navigation-bars-in-beamer
% https://stackoverflow.com/questions/1435837/how-to-remove-footers-of-latex-beamer-templates
\beamertemplatenavigationsymbolsempty
%%% Remover o titulo, autor e data do rodape:
% https://tex.stackexchange.com/questions/113443/remove-author-and-institution-in-footline
\beamertemplatefootempty

%%% Cores das fontes:
% https://en.wikibooks.org/wiki/LaTeX/Colors
\definecolor{ifvermelho}{RGB}{200,12,15}
\definecolor{ifverde}{HTML}{2F9E41}
% Mais cores feitas com http://coolors.co
\definecolor{ifvermelho2}{HTML}{B60B0E} % Nome: International Orange Engineering
\definecolor{ifvermelho3}{HTML}{A50A0D} % Nome: Rufous
\definecolor{ifmeioesq}{HTML}{A2311C} % Nome: Chinese Red
\definecolor{ifmeio}{HTML}{7C5528} % Nome: Coyote Brown
\definecolor{ifmeiodir}{HTML}{567A35} % Nome: Sap Green
\definecolor{ifverde2}{HTML}{42A752} % Nome: Green Pigment
\definecolor{ifverde3}{HTML}{53AF62} % Nome: Medium Sea Green

% https://tex.stackexchange.com/questions/133820/beamer-how-to-change-color-of-infolines-and-frame-title
\setbeamercolor{title}{fg=Red}
\setbeamercolor{frametitle}{fg=Red}
\setbeamercolor{section in head/foot}{fg=White,bg=ifvermelho}
\setbeamercolor{subsection in head/foot}{fg=ifverde}

%%% Cores das caixas de block, exemplo e alerta
% https://tex.stackexchange.com/questions/33231/how-to-change-the-color-of-a-block-within-a-custom-beamer-sty-theme-file
\setbeamercolor{block title}{fg=White,bg=ifverde2}
\setbeamercolor{block body}{fg=White,bg=ifverde3}
% https://tex.stackexchange.com/questions/73163/change-color-scheme-for-example-box-in-beamer
\setbeamercolor{block title example}{fg=White,bg=ifmeio}
\setbeamercolor{block body example}{fg=White,bg=ifmeiodir}
% https://tex.stackexchange.com/questions/96686/change-color-of-itemize-in-beamer-alertblock
\setbeamercolor{block title alerted}{fg=White,bg=ifvermelho3}
\setbeamercolor{block body alerted}{fg=White,bg=ifvermelho2}

%%% Mostrar o sumario novamente antes do inicio da proxima secao
% https://pt.overleaf.com/learn/latex/beamer
\AtBeginSection[]
{\small
  \begin{frame}[plain]
    \frametitle{Sumário}
    \tableofcontents[currentsection]
%	\begin{multicols}{2}
%	\tableofcontents[currentsection]
%	\end{multicols}
  \end{frame}
}

%%% Plano de fundo usado no modelo ODT/DOC/DOCX da Cecom
\usebackgroundtemplate%
{%
    \includegraphics[width=\paperwidth,height=\paperheight]{fundo.jpg}%
}



%%%%%%%%%%%%%%%%%%%%%%%%%%%%%%%%%%%%%%%%%%%%%%%%%%%
%%%%%%%%%%%%%%% Início do documento %%%%%%%%%%%%%%%
%%%%%%%%%%%%%%%%%%%%%%%%%%%%%%%%%%%%%%%%%%%%%%%%%%%
\begin{document}

%%%%% Primeira página do modelo abaixo - PODE SER REMOVIDA, SE DESEJAR %%%%%
%%% Centralizando o titulo da primeira pagina
% https://stackoverflow.com/questions/2365539/centering-titles-when-using-the-beamer-class-in-latex
%%% Removendo as formatacoes do slide
% https://tex.stackexchange.com/questions/281334/left-aligned-title-page-in-beamer-boadilla-usetheme
%{
%\usebackgroundtemplate%
%{%
%    \includegraphics[width=\paperwidth,height=\paperheight]{primeira-pag-menor.jpg}%
%}
%\begin{frame}[plain]{\centerline{\inserttitle}}
%\end{frame}
%}
%%%%% Primeira página do modelo acima - PODE SER REMOVIDA, SE DESEJAR %%%%%

\begin{frame}[plain]{}
    \maketitle
\end{frame}

{\small
\begin{frame}[plain]{Sumário}
    \tableofcontents
%	\begin{multicols}{2}
%	\tableofcontents
%	\end{multicols}
\end{frame}
}

%%%%%%%%%%%%%%%%%%%%%%%%%%%%%%%%%%%%%%%%%%%%%%%%%%%%%%%%%%%%%%%%%%%%%
\section{Introdução}
\subsection{A especificação do trabalho}
\begin{frame}{A especificação do trabalho}
	Esta é a segunda avaliação de 2023/1 de Tópicos em Redes de Computadores:
	\begin{itemize}
		\item Fazer uma aplicação cliente/servidor
		\item Baseada em KVS (Key-Value Store)
		\item Usar TLS (Transport Layer Security)
		\item Demonstrar SIGILO
		\item Demonstrar AUTENTICIDADE
		\item Demonstrar INTEGRIDADE
	\end{itemize}
	Página do componente curricular (disciplina/matéria): \url{https://www.inf.ufpr.br/elias/topredes/}
\end{frame}

\begin{frame}{Entrega}
	O formato de entrega do trabalho:
	\begin{description}
		\item[Relatório:] pode ser vídeo, apresentação ou relatório em texto;
		\begin{itemize}
			\item Optei por vídeo
			\item Outro vídeo com a execução do sistema e comentários
			\item \url{https://youtu.be/rB0gSziJ0dI}
		\end{itemize} \pause
		
		\item[\textit{Logs} de execução:] via demonstrações no vídeo ou \textit{logs} em uma página \textit{web};
		\begin{itemize}
			\item Optei por ambos
			\item Outro vídeo com a execução do sistema e comentários
			\item \url{https://youtu.be/rB0gSziJ0dI}
			\item Página: \url{https://www.inf.ufpr.br/jsouza/}
		\end{itemize} \pause
		
		\item[Código comentado:] entrega do código comentado
		\begin{itemize}
			\item \textbf{Este vídeo}
			\item Página: \url{https://www.inf.ufpr.br/jsouza/}
			\item GitHub: \url{https://github.com/josiney-souza/ufpr-topicos-redes}
		\end{itemize}
	\end{description}
	Entrega por e-mail
\end{frame}

\subsection{A organização do sistema/aplicação}
\begin{frame}{A organização do sistema/aplicação}
	O sistema está assim organizado - \textbf{códigos}:
	\begin{description}
		\item[Linguagem:] Python
		\item[02-cliente.py:] aplicação cliente - entra em contato com o servidor para solicitar serviços ou recursos;
		\item[02-servidor.py:] aplicação servidor - recebe as demandas dos clientes e retorna alguma ação;
		\item[02-invasor.py:] aplicação cliente não autorizada a se comunicar com o servidor;
		\item[confs\_comuns.py:] biblioteca pessoal de funções e configurações comuns aos clientes e ao servidor.
	\end{description}
\end{frame}

%\begin{frame}{}
%	T
%\end{frame}

\section{confs\_comuns.py}
\begin{frame}{confs\_comuns.py}
	Dividido em:
	\begin{itemize}
		\item Variáveis globais de configurações
		\item Funções:
		\begin{itemize}
			\item cripto\_chave\_assim()
			\item descripto\_chave\_assim()
			\item cripto\_rot13()
			\item descripto\_rot13()
			\item alterar()
		\end{itemize}
	\end{itemize}
\end{frame}

\section{02-servidor.py}
\begin{frame}{02-servidor.py}
	Dividido em:
	\begin{itemize}
		\item Funções:
		\begin{itemize}
			\item envia\_menu()
			\item envia\_todo\_banco()
			\item criptografar()
			\item debugar()
		\end{itemize}
		\item Prepara a conexão
		\item Interage com o cliente
	\end{itemize}
\end{frame}

\section{02-cliente.py}
\begin{frame}{02-cliente.py}
	Dividido em:
	\begin{itemize}
		\item Prepara a conexão
		\item Interage com o servidor
	\end{itemize}
\end{frame}

\section{02-invasor.py}
\begin{frame}{02-invasor.py}
	Praticamente o mesmo código do cliente.

	Dividido em:
	\begin{itemize}
		\item Prepara a conexão
		\begin{itemize}
			\item Ou não carrega certificados;
			\item Ou carrega certificados errados.
		\end{itemize}
		\item TENTA interagir com o servidor
	\end{itemize}
\end{frame}

\begin{frame}[plain]{}
    \maketitle
\end{frame}

%\section{Códifo-fonte}

%%%%% Última página do modelo abaixo - PODE SER REMOVIDA, SE DESEJAR %%%%%
%{
%\usebackgroundtemplate%
%{%
%    \includegraphics[width=\paperwidth,height=\paperheight]{ultima-pag-menor.jpg}%
%}
%\begin{frame}[plain]{}
%\end{frame}
%}
%%%%% Última página do modelo acima - PODE SER REMOVIDA, SE DESEJAR %%%%%
	
\end{document}